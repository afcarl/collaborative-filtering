\documentclass[10pt,twocolumn]{article}
\usepackage{hyperref}
\usepackage{url}
\usepackage{graphicx}
\usepackage{cite}
\usepackage{geometry}



\setlength{\voffset}{-0.9in}
\setlength{\headsep}{0pt}



\title{TC1 Project: Know your customers}

\author{Aris Tritas, Laurent Cetinsoy}

\begin{document}

\maketitle

\begin{abstract}

A collaborating filter problem is addressed with the following methods: NMF approach and autoencoders.

\end{abstract}

\section{Introduction}

In collaborative filtering, each item of a collection $n_i$ is rated by several members of a population $n_u$. The aim is to predict the item ratings that a person would give. The usually final goal is to recommend items to a user that he would have rated with a high score. 

In this setup, let $R$ be a $(n * m)$ matrix where $r_{ij}$ is the rating of the $j^{th}$  item from the $i^{th}$ user.

\subsection{Non negative matrix factorization}

In non negative factorizations the rating product matrix is factorized to $R = UV$ where coefficients of U and V are non negatives. 

In order to find such matrix a reconstruction error is minimized by gradient descent : 

$$ argmin_{U,V} ( ||R - UV||_{non negative coefficients} ) $$ 

The reconstruction error is computed over all non negative coefficient of the R matrix. In other words, the algorithm try to recover the available ratings only. 

Comment on est sur que les coefficients soient positifs ?

\subsection{ALS-WR}

The ALS-WR algorithm \cite{zhou2008large} differs from NMF for two main reasons: a regulirazation term and the fact that the gradient descent is done by alternatively updating U and fixing V then, updating V and fixing U. 



\subsection{Auto-encoders}

Auto encoder is neural network (NN) approach used in unsupervised learning for dimensionality reduction. Used in the supervised classification task NN aimes at lowering a prediction error over a training set. Auto encoder aimes at lowering the reconstruction error, ie Lowering $$ || X - NN(X) || $$. 

We used the approach proposed by \cite{strub2016hybrid} to our collaborating filter problem. 

\section{Result}

\begin{tabular}{|l|c|r|}
  \hline
  NMF & ALS-WR & AE \\
  \hline
  1.1 & 1.2 & 1.3 \\
  
  \hline
\end{tabular}


\section{Conclusion}

[1] Yunhong Zhou, Dennis Wilkinson, Robert Schreiber and Rong Pan - Large-scale Parallel Collaborative Filtering for the Netflix Prize.

\bibliographystyle{plain}
\bibliography{biblio} % mon fichier de base de données s'appelle bibli.bib


\end{document}