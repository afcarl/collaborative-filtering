\documentclass[10pt,twocolumn]{article}
\usepackage{hyperref}
\usepackage{url}
\usepackage{graphicx}
\usepackage{cite}
\usepackage{geometry}
\usepackage{amsmath}
\setlength{\voffset}{-0.9in}
\setlength{\headsep}{0pt}

\DeclareMathOperator*{\argmin}{arg\,min}

\title{TC1 Project: Know your customers}

\author{Aris Tritas \and Laurent Cetinsoy}

\begin{document}

\maketitle

\begin{abstract}

We are given a training set describing the ratings assigned by customers to items (e.g a firm's services). 
A collaborating filter problem is addressed with the following methods: NMF approach and autoencoders.

\end{abstract}

\section{Introduction}

In collaborative filtering, each item $i$ of a collection is rated by one or more members of a population $n_u$. The aim is to predict the rating that a person would give to an unseen item, so as to be able to recommend new items to users such that it fits their preferences with high probability.

In this setup, let $R$ represent a $n \times m$ matrix. The rating of the $j^{th}$ item from the $i^{th}$ user is $r_{ij}$.

\subsection{Data}
Dataset size: 2536705 examples in the training set, 1306637 in the test set. Total number of users 93705 (of which only 92088 are in the training set) and 3561 items. The dataset is very sparse,  less than $1\% $ of the ratings matrix is filled.
Ratings distribution: [114214, 251596, 728518, 860463, 581913] In proportion of the total number of ratings: [0.0450, 0.0991, 0.2872, 0.3392, 0.2293].

\subsection{Hidden Markov Model}
It is possible to model the problem of sequential prediction using a hidden Markov model. In order to do that efficiently, it is assumed that 
\subsection{Non negative matrix factorization}

In non negative factorizations we seek SPD matrices U and V such that the rating product matrix can be factorized as $R = UV^T$. It is possible to find such matrices that minimize the reconstruction error defined below by using an iterative optimization method (e.g gradient descent): 

\begin{equation*}\begin{split}
\argmin_{U,V} & \frac{1}{2} ||R - UV||^2 \\
			  + & \alpha \lambda_1 (||U||_1 + ||V||_1) \\
			  + & \frac{1}{2} \alpha \lambda_2  (||U||^2 + ||V||^2) 
\end{split}\end{equation*}  

The reconstruction error is computed over all non negative coefficients of the R matrix. In other words, the algorithm only tries to recover the available ratings. 

\subsection{ALS-WR}

The Alternating Least Squares with Regularization algorithm \cite{zhou2008large} differs from NMF for two main reasons: a regularization term and the fact that the minimization is done by alternatively updating U while keeping V fixed and then updating V while keeping U fixed.

The parameters used were $\lambda = 0.01$ and a latent dimension $d = 100$.


\subsection{Auto-encoders}

The auto-encoder is a neural network approach used in unsupervised learning for dimensionality reduction. It aims at lowering the reconstruction error, which can be written as:
$$ || X - NN(X) || $$. 

In the supervised classification task, the auto-encoder's objective is to lower a prediction error over a training set.

The approach proposed by \cite{strub2016hybrid} was used to tackle our collaborating filter problem. Two different auto-encoders can be trained depending on whether the users or the items are considered as training cases. 

\section{Result}
The metric used to evaluate the performance of a training method for this type of problem is either the RMSE (root mean squared error) or the MAE (mean absolute error). The scores below were computed on a validation set selected uniformly at random amongst the training set:
\begin{tabular}{|l|c|c|r}
  \hline
  NMF & ALS-WR & User AE & Item AE \\
  \hline
  3.70 & 0.934 & 0.913 & 0.938 \\
  \hline
\end{tabular}


\section{Conclusion}

[1] Yunhong Zhou, Dennis Wilkinson, Robert Schreiber and Rong Pan - Large-scale Parallel Collaborative Filtering for the Netflix Prize.

\bibliographystyle{plain}
\bibliography{biblio} % mon fichier de base de données s'appelle bibli.bib


\end{document}